\subsection{Критерии сравнения}
Могут быть выделены следующие критерии сравнения соответствующих алгоритмов:

\subsubsection{Тип криптографического алгоритма}
Как было показано выше, существует большое разнообразие типов криптографических алгоритмов. При выборе какого-то из них для тех или иных целей необходимо учитывать математические аспекты его реализации, потому что вопрос о надёжности математической базы является дискуссионным\cite{lattice_attack}, и по причине постоянного развития в сфере криптоанализа нельзя с полной уверенностью утверждать, что какая либо из криптосистем не будет скомпрометирована в дальнейшем.

\subsubsection{Заявляемая стойкость криптографического примитива/реальная стойкость криптографического примитива}

Для оценки качественных свойств криптографических свойств алгоритмов, NIST ввёл несколько уровней стойкости по критерию количества вычислительных ресурсов, необходимых для поиска ключа, с аналогичной характеристикой для указанной в уровне симметричной криптосистемой\cite{Security_Evaluation_Criteria}:
\begin{enumerate}
    \item Любая атака, соответствующая 1-у уровню безопасности, должна требовать вычислительных ресурсов, сравнимых или превышающих те, которые требуются для поиска ключа в блочном шифре со 128-битным ключом (например, AES128).
    \item Любая атака, соответствующая 2-у уровню безопасности, должна требовать вычислительных ресурсов, сравнимых или превышающих те, которые требуются для поиска коллизий в 256-битовой хэш-функции (например, SHA256/ SHA3-256).
    \item Любая атака, соответствующая 3-у уровню безопасности, должна требовать вычислительных ресурсов, сравнимых или превышающих те, которые требуются для поиска ключа в блочном шифре со 192-битным ключом (например, AES192).
    \item Любая атака, соответствующая 4-у уровню безопасности, должна требовать вычислительных ресурсов, сравнимых или превышающих те, которые требуются для поиска коллизий в 384-битовой хэш-функции (например, SHA384/ SHA3-384).
    \item Любая атака, соответствующая 5-у уровню безопасности, должна требовать вычислительных ресурсов, сравнимых или превышающих те, которые требуются для поиска ключа в блочном шифре с 256-битным ключом (например, AES 256).
\end{enumerate}

В настоящий момент в качестве эквивалента для сравнения уровня стойкости криптографических алгоритмов используют так называемую битовую стойкость, вычисляемую как двоичный логарифм трудоемкости алгоритма нарушения свойств безопасности. В США существуют три базовых уровня стойкости, равные 128, 192 и 256 бит, что равняется возможным длинам ключа американского стандарта шифрования AES\cite{Security_Evaluation_Criteria}. 

Данный параметр станет одним из ключевых, при изучении режимов работы рассмотренных алгоритмов.

\subsubsection{Модель нарушителя в которых обосновывается стойкость алгоритма}
Существует ряд стандартных моделей угроз и нарушителя, для которых происходит обоснование безопасности криптографических алгоритмов. 

В перспективе внедрения криптографического алгоритма необходимо будет учитывать модель угроз и нарушителей системы, в которую выбранный алгоритм будет внедряться.

\subsubsection{Длина открытого ключа/длина секретного ключа/длина подписи (для алгоритмов подписи), длина шифртекста (для алгоритмов шифрования) и т.п.}
Эксплуатационные характеристики, влияющие на количество требуемой для реализации памяти и/или влияющие на количество бит информации, необходимых для пересылки/хранения.

Данная характеристика может стать решающей при прочих равных, так как сетевые устройства и устройства систем умного дома не всегда могу располагать необходимыми вычислительными ресурсами для внедрения нового алгоритма, что может значительно затруднить его распространение и повысит риски.

\subsubsection{Скорость реализации алгоритмов}
Генерации ключей, алгоритмов шифрования/расшифрования, алгоритмов вычисления/проверки подписи.

Скорость работы необходимо учитывать, так как это непосредственно будет влиять на скорость интегрирования и распространения нового решения, величину занятых ресурсов и общий пользовательский опыт. Алгоритмы с большей сложностью вычислений могут потребовать значительного увеличения вычислительных ресурсов, что не является благоприятной перспективой и также может затруднить их интегрирование в сетевые устройства и устройства умного дома.