\section{Термины и определения}

\textbf{Алгоритм зашифрования} — алгоритм криптографический, реализующий функцию зашифрования. В случае шифрсистем блочных получается использованием алгоритма зашифрования блочного базового в конкретном режиме шифрования.

\textbf{Алгоритм криптографический} — алгоритм, реализующий вычисление одной из функций криптографических.

\textbf{Алгоритм расшифрования} — алгоритм криптографический, обратный к алгоритму зашифрования и реализующий функцию расшифрования.

\textbf{Алгоритм шифрования} — под алгоритмом шифрования в зависимости от контекста понимается алгоритм зашифрования или алгоритм расшифрования.

\textbf{Анализ криптографический} — исследование системы криптографической с целью получения обоснованных оценок ее стойкости криптографической. Результаты а. к. могут использоваться разработчиком и пользователем законным криптосистемы для оценки эффективности системы защиты информации от потенциального противника и/или нарушителя, а потенциальным противником и/или нарушителем для подготовки и реализации атаки на криптосистему. А. к. проводится путем исследования криптосистемы, а также моделирования (выполнения) различных атак на криптосистему.

\textbf{Аутентификация сообщения} — проверка того, что сообщение было получено неповрежденным, неизмененным (с момента отправления), то есть проверка целостности. Если стороны доверяют друг другу, то а. с. осуществляется применением системы имитозащиты. Для не доверяющих друг другу сторон необходимо использовать систему подписи цифровой.

\textbf {Дешифрование} — процесс аналитического раскрытия противником и/или нарушителем сообщения открытого без предварительного полного знания всех элементов системы криптографической. Если этот процесс поддается математической формализации, говорят об алгоритме дешифрования.

\textbf{Система криптографическая} — система обеспечения безопасности информации криптографическими методами в части конфиденциальности, целостности, аутентификации, невозможности отказа и неотслеживаемости. В качестве подсистем может включать системы шифрования, системы идентификации, системы имитозащиты, системы подписи цифровой и др., а также систему ключевую, обеспечивающую работу остальных систем. В основе выбора и построения с. к. лежит условие обеспечения стойкости криптографической.

\textbf{Хеш-значение (hash-code)} —значение хеш-функции для данного аргумента.

\textbf{Хеш-функция} — функция, отображающая входное слово конечной длины в конечном алфавите в слово заданной, обычно фиксированной длины.

\textbf{Хеш-функция криптографическая}  — хеш-функция, сочетающая в себе свойства хеш-функции односторонней, хеш-функции с прообразами вторыми трудно обнаружимыми и хеш-функции с коллизиями трудно обнаружимыми. Особо выделяют хеш-функции криптографические, задаваемые ключом, имеющие другое содержание.

\clearpage
