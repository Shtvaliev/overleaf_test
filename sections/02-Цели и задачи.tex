\section{Цели работы, поставленные задачи}

Задачей данного обзора является рассмотрение наиболее перспективных направлений развития алгоритмов постквантовой криптографии и основных их представителей. В частности:



\begin{itemize}
    \item Коды, исправляющие ошибки, и криптосхема Мак-Элиса;
    \item Целочисленные решётки и криптосхема CRYSTALS-Kyber;
    \item Скрытые уравнения поля и криптосхема GeMSS;
    \item Хэш-функции и криптосхема SPHINCS$^+$
\end{itemize}

Мы выбрали данные криптоагоритмы так как они прошли 3 раунд конкурса NIST (National Institute of Standards and Technology) по выбору квантово-устойчивых криптоалгоритмов. Конкурс стартовал в 2017 году и был нацелен на выбор алгоритмов постквантовой криптографии, пригодных для выдвижения в качестве стандартов. 

Мы сосредоточимся на анализе математических основ данных алгоритмов, их применимости в различных сценариях, а также оценке эффективности их реализаций.
\clearpage