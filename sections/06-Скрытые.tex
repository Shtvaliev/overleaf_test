\subsection{Скрытые уравнения поля}

Скрытые уравнения поля (Hidden Field Equations) — вид криптографической системы с открытым ключом. Данная система является обобщением системы Матцумото-Имаи \cite{matsuo-imai} и впервые была представлена Жаком Патарином в 1996 году \cite{HFE}. 

Большинство криптосистем с открытым ключом, применяемых на практике, опираются на целочисленную факторизацию или дискретные логарифмы (в конечных полях или на эллиптических кривых). Однако такие системы сталкиваются с двумя основными недостатками:

\begin{enumerate}
    \item Исследования в области теории чисел снижают эффективность вычислений, что требует увеличения размеров параметров для обеспечения безопасности. 
    \item В случае создания достаточно мощных квантовых компьютеров, алгоритм Шора \cite{Shor} делает текущие системы полностью уязвимыми. Поэтому крайне важно искать альтернативные, неуязвимые системы.
\end{enumerate}

Устойчивость криптосистем, основанных на скрытых уравнениях поля, опирается на сложности решения системы многомерных квадратичных уравнений над конечным полем, что является NP-полной задачей. Эти системы характеризуются высокой скоростью и небольшими требованиями к вычислительным ресурсам, однако требуют больших длин открытых ключей. Примером такой криптосистемы является HFE (Hidden Fields Equations), основанная на скрытых уравнениях поля.

Давайте разберем алгоритм работы первой версии HFE. Обозначим через $F$ конечное поле мощности $q$ и характеристики $p$ для некоторой простой степени $q = p ^ m$. Пусть $F_{q^n}$ - расширение степени $n$ из $F_q$. Пусть многочлен $x$ над $F_{q^n}$ степени $d$ для целых чисел для выражения $\theta_{ij}, \varphi_{ij}, \xi_k \geqslant 0 $. Выражении имет вид:
\begin{equation}
    f(x) = \sum_{i,j} \beta_{i,j}x^{q^{\theta_{ij}} + q^{\varphi_{ij}}} + \sum_{k} \alpha_k x^{q^{\xi_k}} + \mu \in F_{q^n} [x]
\end{equation}

Так как $Fq$ изоморфен $F_q,[x]/(g(x))$, если $g(x) \in F_q[x]$ неприводимо степени $n$, элементы $F_{q^n}$ могут быть представлены в виде n-кратных над $F_{q^n}$, а $f$ может быть представлен в виде многочлена от $n$ переменных $x_1,x_2,\ldots,x_n$ над $F_q$.
\begin{equation}
    f(x_1,\ldots,x_n) = (p_1(x_1,\ldots,x_n),\ldots,p_n(x_1,x_2,\ldots,x_n)) \in F_q [x_1,\ldots,x_n]
\end{equation}

Как мы можем видеть, $p_i(x_1,\ldots,x_n) \in F_q[x_1,\ldots,x_n]$, для $i=1,2,\ldots,n$. Так как $p_i$ являются квадратичными многочленами из-за выбора $f$ и того факта, что $x \longmapsto x^q$ является линейной функцией от $F_{q^n} \rightarrow F_{q^n}$.

Давайте предположим, что исходные данные представлены в виде n-кратного $x$ над $F_q$, где $F_q$, общеизвестно.  (Таким образом, если $p = 2$, исходные данные представлены $nm$ битами). Более того, мы предполагаем, что некоторая избыточность была включена в представление каждого сообщения таким образом, что избыточность нелинейно зависит от $M$. Хороший способ сделать это - использовать код, исправляющий ошибки. Если $p = 2$, мы могли бы также получить $x$, объединив двоичное представление $M$ и первые 64 бита $h(M)$, где $h$ - хэш-функция, такая как MD5 или SHA, при условии, что результирующий $z$ содержит не более $nm$ бит.

Пусть $s$ и $t$ - две аффинные биекции $(F_q)^n \rightarrow (F_q)^n$, где $(F_q)^n$ рассматривается как $n$-мерное векторное пространство над $F_q$. Как $s$, так и $t$ могут быть представлены в виде $n$-кратных многочленов от $n$ переменных над $F_q$ общей степени 1. Используя функцию (6) и некоторое представление $F_{q^n}$ поверх $F_q$, как мы рассматривали ранее, функция $(F_q)^n \rightarrow (F_q)^n$, которая присваивает $t(f(s(x)))$ до $x \in (F_q)^n$ может быть записана как
\begin{equation}
    t(f(s(x_1,\ldots,x_n))) = (p_1(x_1,\ldots,x_n),\ldots,p_n(x_1,x_2,\ldots,x_n)) \in F_q [x_1,\ldots,x_n]
\end{equation}
с $p_i(x_1,\ldots,x_n) \in F_q[x_1,\ldots,x_n]$, для $i=1,2,\ldots,n$. Переменная $p_i$ являются квадратичными многочленами из-за выбора $s, t$ и $f$. Кроме того, учитывая $s, t, f$ и способ представления $F_{q^n}$ поверх $F_q$, многочлены $p_i$ могут быть эффективно вычислены. Обратное, однако, кажется трудным, если правильно выбраны $s, t$ и $f$. 

\begin{comment}
Это приводит к следующей схеме шифрования с открытым ключом, которая завывается «уравнениями скрытого поля» (HFE).

\paragraph{Закрытый ключ.} Функция $f$, как мы рассматривали ранее, содержит две аффинные биекции $s$ и $t$, как указано выше, и некоторый способ представления $F{q^n}$ поверх $F_q$. Последний может быть закрытым, а может и не быть, поскольку изменение представления эквивалентно изменению $s$ или $t$; следовательно, мы можем предположить некоторый фиксированный способ представления $F{q^n}$.

\paragraph{Открытый ключ.} Многочлены $p_i$ для $i= 1,2,\ldots,n$ как указано выше, вычисляются с использованием секретного ключа $f , s, t$. Кроме того, $F_q$, степень расширения $n$ и способ добавления избыточности к сообщению является открытым.

\paragraph{Зафрование.} Чтобы зашифровать n-кратный $x = (x_1,\ldots,x_n) \in (F_q)^n$ (представляющий сообщение M плюс избыточность), вычислите зашифрованный текст по формуле:
\begin{equation}
    y=(p_1(x_1,\ldots,x_n),\ldots,p_n(x_1,x_2,\ldots,x_n)).
\end{equation}

\paragraph{Расшифрование.} Чтобы расшифровать зашифрованный текст $y$, сначала найдите все решения $z$ уравнения $f(z) = t^{-1}(y)$, затем вычислить все $s^-1(z)$ и, наконец, воспользуемся избыточностью, чтобы найти M из них.
\end{comment}

Скрытые уравнения поля представляют собой четыре основные модификации которые рассматриваются в статье \cite{hfe_variations}: "плюс", "минус", "v" и "f", они могут быть комбинированы в различных вариантах:

\begin{enumerate}
    \item Модификация "плюс" включает в себя линейное комбинирование уравнений с некоторыми случайными уравнениями.
    \item Модификация "минус", предложенная Ади-Шамиром, устраняет избыточность "r" из уравнений.
    \item Модификация "f" включает фиксацию определенных входных переменных "f" открытого ключа.
    \item Модификация "v" определяется как сложная конструкция, при условии, что обратная функция может быть найдена только в том случае, если определенные переменные "v" зафиксированы.
\end{enumerate}

\subsubsection{GeMSS}
GeMSS (Great Multivariate Short Signature) - это схема подписи, которая следует парадигме hash-and-sign с применением итераций Фейстеля-Патарина. GeMSS использует функцию-ловушку, основанную на уравнении скрытого поля с уксусными (vinegar) переменными и минус (minus)-модой (HFEv-).

GeMSS относится к семейству многомерных криптосистем с большим полем. Основная идея этих схем заключается в использовании биективного отображения между $GF(q^n)$ и $GF(q)^n$ таким образом, что многомерная функция ловушки (выраженная в терминах малого поля $GF(q))$ может быть выражена как одномерная функция над большим полем $GF(q^n)$. В таком виде функция может быть эффективно инвертирована. Для получения открытого ключа функция складывается с линейными картами над малым полем. Криптосистема HFE была предложена после того, как оригинальная схема большого поля Мацумото и Имаи была нарушена в \cite{destroyed-matsuo-imai}. Однако HFE с безопасными параметрами имеет очень медленное подписание. Модификаторы "уксусные" и "минус" были добавлены в \cite{128-bit-long-digital-signatures} в попытке повысить стойкость при незначительном снижении производительности.
Основными параметрами криптосистемы являются значения:
\begin{itemize}
    \item D - целое положительное число, являющееся степенью секретного многочлена. D таково, что $D = 2^i$ при $i \geq 0,$ или $D=2^i+2^i$  при  $i \neq j,$ при $i,j \geq 0$; 
    \item K - размер выходного сигнала в битах хэш-функции;
    \item $\lambda$ -уровень защищенности системы GeMSS;
    \item m - количество уравнений в открытом ключе;
    \item $nb_{ite} > 0$, количество итераций в процессах верификации и подписи;
    \item n - степень расширения поля $F_2$;
    \item v - количество уксусных переменных;
    \item $\delta$ - число минусов (число уравнений в открытом ключе таково, что равно $m = n-\delta$).
\end{itemize}
\paragraph{Закрытый и открытый ключ.} Открытым ключом в GeMSS является набор \\ $p1,\ldots,pm$  $F_2[x_1,\ldots,x_n+v]$ из $m$ квадратичных уравнений в $n + v$ переменных. Эти уравнения являются производными от многомерного полинома $F \in F_{2^n} [X, v_1, . . . . , v_v]$, имеющего определенную форму, описанную в (4), так что построение сигнатуры, по сути, эквивалентно нахождению корней $F$.

\paragraph{Закрытый ключ.} Он состоит из пары невырожденных матриц $(S, T) \in GL_n+v$ $(F_2) \times GL_n (F_2)$ и многочлена $F \in F_{2^n} [X, v_1, . . . . , v_v]$ со следующей структурой:

\begin{equation}
\sum_{\substack {0 \leqslant j< i < n \\ 2^i+2^j \leqslant D}} A_{i,^j} X^{2^2+2^j} + \sum_{\substack {0 \leqslant 0 \leqslant i < n \\ 2^i \leqslant D}} \beta_i (v_1,\ldots,v_v) X^{2^i} + \gamma (v_1,\ldots,v_v),
\end{equation}

где $A_{i,j} \in F_{2^n},\forall i,j,0 \leqslant j < i < n$, каждый $\beta_i : F^v_2 \to F_{2^n}$ является линейным и $\gamma (v_1,\ldots,v_v) : F_{v^2} \to F^v_2$ является квадратичным. Переменные $v_1,\ldots, v_v$ называются уксусными переменными. Многочлен $F \in F_{2^n} [X,v_1,\ldots,v_v]$ вида (4) имеет HFEv- форму.

\paragraph{Открытый ключ.} Он задается набором из $m$ квадратичных беcквадратичных нелинейных полиномов от $n + v$ переменных над $F_2$. То есть открытый ключ - это $p = (p_1, . . . , p_m) \in F_2[x_1, . . . , x_{n+v}]^m$. Он получается из секретного ключа путем взятия первых $m = n - \delta$ полиномов от:

\begin{equation}
\bigg ( f_1((x_1,\ldots,x_{n+v})S),\ldots,f_n((x_1,\ldots,x_{n+v})S) \bigg ) T
\end{equation}

и сокращая его по модулю уравнений поля, т.е. по модулю $|x^2_1 - x_1, . . . , x^2_{n+v} - x_{n+v}|$. Обозначим эти полиномы через $p = (p_1,\ldots,p_m) \in F_2[x_1,\ldots,x_{n+v}]^m$.

В качестве входных данных используется параметр безопасности $\lambda$. уровень стойкости GeMSS зависит от $D, n, v, m$ $\lambda \in \{128, 192, 256\}$. 

\paragraph{Процесс подписи.} Основной этап процесса подписи требует решения:

\begin{equation}
p_1(x_1,\ldots,x_{x+v})-d_1=0,\ldots,p_m(x_1,\ldots,x_{n+v})-d_m=0
\end{equation}
для $d=(d_1,\ldots,d_m) \in F_2^m$.

Для этого произвольно выбираем $r = (r_1,\ldots,r_{n-m}) \in F_{n-m}$ и добавляем его к $d$. Это дает $d' = (d,r) \in F^n_2$. Далее вычисляем $D' = \varphi^{-1}(d' \times T^{-1}) \in F_{2^n}$ и пытаемся найти корень $(Z,z_1,\ldots,z_v) \in F_{2^n} \times F^v_2$.
$F_{2^n} \times F^v_2$ многомерного уравнения:

\begin{equation}
F(Z,z_1,\ldots,z_v)-D' =0
\end{equation}

Для решения этого уравнения произвольно выбираем $v \in F^v_2$ и рассматриваем одномерный многочлен $F(X,v) \in F_{2^n}[X]$. Далее находим корни одномерного уравнения:

\begin{equation}
F(X,v)-D'=0
\end{equation}

Основной частью генерации сигнатуры является вычисление корней из $F_{D'}(X)=F(X,v) - D'$. Для этого мы используем алгоритм Берлекампа \cite{Berlecamp}.

Сложность алгоритма определяется следующим общим результатом:
Пусть $F_q$ - конечное поле, а $Mq (D)$ - число операций в $F_q$ для умножения двух многочленов степени $\leqslant D$. Учитывая $f \in F_q[x]$ степени $D$, мы можем найти все корни из $f$ над $F_q$, используя:
\begin{equation}
O \Big ( M_q(D) \log (D) \log (Dq)  \Big )
\end{equation}
Пусть $d \in F_m$ и $s \leftarrow Inv_p(d,sk = (F,S,T)) \in F^{n+v}_2$. Мы имеем $p(s) = d$, где $p$, связанно с $sk$.