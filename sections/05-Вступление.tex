\section{Вступление}
В настоящее время абсолютное большинство информационных систем передачи данных используют на одном из этапов шифрования асимметричные криптосистемы с открытым ключом. Большой проблемой такого подхода является уязвимость современных алгоритмов асимметричного шифрования в обозримом будущем.

По оценками специалистов компании QApp \cite{quantumthreat}, к 2028 году возможно появление достаточно мощного квантового компьютера для реализации на нём алгоритма Шора \cite{Shor} и решения задач факторизации и дискретного логарифмирования, лежащих в основе используемых на данный момент криптосистем, используемых в таких протоколах шифрования данных в сети, как TLS или IKE.

В связи с возрастающей угрозой необходимо уже сейчас использовать для передачи данных криптографические алгоритмы достаточно стойкие, чтобы в перспективе противостоять криптоанализу с применением квантового компьютера. В данной статье приведён обзор наиболее перспективных направлений развития постквантовой криптографии и описаны их представители.

\clearpage
